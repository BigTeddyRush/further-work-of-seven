\documentclass[german,version-2020-11]{uzl-thesis}

\UzLThesisSetup{
  %
  Logo-Dateiname        = {uzl-thesis-logo-itcs.pdf},
  Verfasst              = {am}{Institut für Software Engineering und Programming Languages},
  %
  % The titles:
  %
  Titel auf Deutsch     = {
    Können semantische Ähnlichkeiten von Wörtern die Schlussfolgerungen des gesunden Menschenverstands verbessern?  Eine Fallstudie mit Prover E und SUMO.
  }, 
  Titel auf Englisch    = {
    Can semantic similarities of words enhance common sense reasoning?  A case study with prover E and SUMO.
  },
  %
  % Author and supervisor:
  % 
  Autor                 = {Julian Britz},
  Betreuerin            = {Prof. Dr. Diedrich Wolter},
  % 
  % Optional: Supporting persons and institutions. The text should be
  % in German, even for an English thesis.
  %
  Mit Unterstützung von = {Moritz Bayerkuhnlein},
  %
  Bachelorarbeit,
  Studiengang           = {Informatik},
  %
  % Date on which the thesis is turned in German, formatted the
  % traditional German way:
  %
  Datum                 = {06. Juli 2025},
  %
  % The English abstract. You must always provide abstracts in German
  % and in English. 
  %
  Abstract              = {
    Abstract.
  },
  Zusammenfassung       = {
    Zusammenfassung.
  },
  %
  Alphabetische Bibliographie,
  % Alternatively:
  % Numerische Bibliographie
}
%
\UzLStyle{alegrya modern design}
%%%%%%%%
%
% Now, include the package you need here using \usepackage. 
%
% However, many standard packages are already loaded by the class:
%
% amsmath, amssymb, amsthm, babel, biblatex, csquotes, etoolbox,
% filecontents, fontspec, geometry, hyperref, tikz (with libraries
% arrows.meta, positioning and shapes), varioref, url 
%
% Indeed, in many cases you will not need any extra packages.
%
%%%%%%%





\begin{document}

%
% The title page and table of contents will be inserted automatically
% here. 
%


\chapter{Einleitung}
% In a German thesis write: \chapter{Einleitung}
  \begin{itemize}
    \item Bedeutung der logischen Schlussfolgerung im Bereich KI und der natürlichen Sprachverarbeitung
    \item Typische Ansätze
    \item Potenzial von semantischen Informationen zur Verbesserung der Auswahl von Axiomen
    \item Beschreibung von E als ein effektiver Theorembeweiser für die Aussagenlogik
    \item Adimen-SUMO als komplexe Wissensbasis für die Simulation und Bewertung von Schlussfolgerungsstrategien
    \item Ziel und Beitrag der Arbeit
  \end{itemize}
\chapter{Verwandte Arbeit}
%
\chapter{Vorwissen}
  \section{Commen sense reasoning}
    \begin{itemize}
      \item Erklärung
      \item Herausforderungen
    \end{itemize}
  \section{Word Embeddings}
    \begin{itemize}
      \item Definition
      \item Erstellung
      \item Eigenschaften
      \item Anwendung
      \item Limitierung und Herausforderungen
    \end{itemize}
  \section{Grammatiken}
    \begin{itemize}
      \item Erklärung. Was sind Grammatiken und welche gibt es?
      \item Aufbau und Struktur
      \item SUMO
    \end{itemize}
  \section{Theorembeweiser}
    \begin{itemize}
      \item Erklärung. Was sind Theorembeweiser und welche gibt es?
      \item Funktionalität
      \item Grenzen und Herausforderungen
      \item Prover E
        \begin{itemize}
          \item Auto mode 
          \item Satauto mode 
        \end{itemize}
    \end{itemize} 
\chapter{Selektionsstrategien}
  \begin{itemize}
    \item Syntaktisch 
    \item Semantisch
    \item Kombination
  \end{itemize}
\chapter{Experimente}
  \begin{itemize}
    \item Standard vs. Satauto vs. Auto
    \item SInE vs. SeVen 
    \item Welche Axiome werden gewählt?
    \item Statistiken
      \begin{itemize}
        \item Mean variable count
        \item Count signs
        \item Character Count
        \item Variable appearence
        \item Proofs found in first named
        \item Time to find proof
        \item Summarized time proof found
        \item Conclusion
      \end{itemize}
    \item Add Axiome
      \begin{itemize}
        \item 1000 häufigste
        \item SInE Strategie als Auswahl
      \end{itemize}
    \item Vampire
    \item Conclusion
  \end{itemize}
\chapter{Weiterführende Arbeit}
\begin{itemize}
  \item Wahl der richtigen Axiome durch neuronales Netz
\end{itemize}
\chapter{Quellenverzeichnis}
\end{document}